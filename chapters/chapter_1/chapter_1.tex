\chapter{Einleitung}
Text\,\footnote{eine Fu{\ss}note}


\section{Fragestellung}

Aufbau eines Lautsprechers nach der Exciter-Technologie auf einer Sandwichplatte. Messtechnische Erfassung aller 
erforderlichen Parameter für den Vergleich zu einem Lautsprecher mit Lautsprechermembran (Ausarbeitung der Vor-und 
Nachteile).

\section{Motivation und Zielsetzung}

Ziel dieser Bachelorarbeit ist die Entwicklung eines flächigen Soundsystems auf Basis von Exciter-Körperschallwandlern, das
für den Einsatz im Akustiklabor der Hochschule Ostfalia vorgesehen ist. Die Aufgabenstellung umfasst dabei die vollständige
ingenieurwissenschaftliche Bearbeitung des Systems von der konzeptionellen Auslegung über die rechnergestützte Simulation und 
konstruktive Umsetzung bis hin zur experimentellen Erprobung und abschließenden Bewertung. Das entwickelte Soundsystem soll 
nicht nur funktional, sondern auch praxisnah einsetzbar sein und den besonderen Anforderungen des vorgesehenen Einsatzortes 
gerecht werden. Im Mittelpunkt der Arbeit steht die Verwendung von Excitern, die mechanische Schwingungen direkt in flächige 
Strukturen einleiten und diese zur Schallabstrahlung nutzen. Im Gegensatz zu konventionellen Lautsprechersystemen fungiert 
hierbei die angeregte Platte selbst als Schallquelle. Daraus ergibt sich eine enge Kopplung zwischen mechanischem 
Schwingungsverhalten und akustischer Abstrahlung, die eine fundierte Betrachtung sowohl strukturmechanischer als auch 
akustischer Aspekte erfordert. Ein wesentliches Ziel der Arbeit besteht daher darin, diese Zusammenhänge systematisch zu 
untersuchen und für die gezielte Auslegung des Systems nutzbar zu machen.
Die geometrischen Randbedingungen des Soundsystems sind durch den geplanten Einsatz im Akustiklabor vorgegeben. Die 
Abmessungen der Platten sind an das A1-Format gebunden, da diese zusätzlich als Trägermedium für Informationsmaterial und 
Plakate dienen sollen. Dadurch ergeben sich besondere Anforderungen an die mechanische Stabilität, die 
Oberflächenbeschaffenheit sowie die Befestigungsmöglichkeiten der Platten, ohne die akustischen Eigenschaften negativ zu 
beeinflussen. Die doppelte Funktion als Informationsträger und Schallquelle stellt einen zentralen Aspekt der 
Aufgabenstellung dar und erfordert eine sorgfältige konstruktive und materialtechnische Auslegung.
Ein weiterer Schwerpunkt der Arbeit liegt auf der Auswahl und Bewertung geeigneter Plattenmaterialien. Die Materialfindung 
ist entscheidend für das Schwingungs- und Abstrahlungsverhalten des Systems und beeinflusst maßgeblich das resultierende 
Hörerlebnis. Ziel ist es, ein Material beziehungsweise einen Materialaufbau zu identifizieren, der eine gleichmäßige 
Schwingungsanregung, eine ausreichende Schallabstrahlung sowie eine angenehme akustische Wahrnehmung ermöglicht. Dabei sind 
sowohl mechanische Eigenschaften wie Dichte, Elastizitätsmodul und innere Dämpfung als auch praktische Gesichtspunkte wie 
Verfügbarkeit, Verarbeitbarkeit und Robustheit zu berücksichtigen.
Zur Bearbeitung der Aufgabenstellung werden numerische Simulationsmethoden eingesetzt, um das Schwingungsverhalten der 
Platten sowie deren akustische Abstrahlung vorab zu analysieren. Aufbauend auf den Simulationsergebnissen erfolgt die 
konstruktive Auslegung des Systems und die Realisierung eines Prototyps. Dieser wird anschließend messtechnisch untersucht 
und hinsichtlich seiner akustischen Eigenschaften bewertet. Die gewonnenen Messergebnisse dienen sowohl der Validierung der 
Simulationen als auch der abschließenden Beurteilung der Eignung des Systems für den vorgesehenen Einsatz im Akustiklabor.
Insgesamt verfolgt diese Bachelorarbeit das Ziel, ein funktionsfähiges, didaktisch nutzbares und akustisch überzeugendes 
Exciter-Soundsystem zu entwickeln, das sowohl technische als auch praktische Anforderungen erfüllt und einen Mehrwert für den 
Einsatz im Hochschulumfeld bietet.

\subsection{\"Uberschrift 2}

\footnote{eine Fu{\ss}note}



\subsubsection{\"Uberschrift 3}







\begin{enumerate}
	\item Punkt 1
	\item Punkt 2
	\item Punkt 3
	\begin{itemize}
		\item Unterpunkt 3.1
		\item Unterpunkt 3.2
	\end{itemize}
	\item Punkt 4
\end{enumerate}
\subsection{Abbildungen}
...Abbildung\,\ref{fig:excample1} zeigt ein Beispiel.


\begin{figure}[h!]
	\centering
		\includegraphics[width=0.5\linewidth]{pictures/picture_excample.pdf}
	\caption{Beispiel\,\cite{example}}
	\label{fig:excample1 Was passiert jetzt?}
\end{figure}

\begin{figure}[h!]
\centering
\begin{minipage}{0.5\linewidth}
\includegraphics[width=1.0\linewidth]{pictures/picture_excample.pdf}%
\end{minipage}
\begin{minipage}{0.25\linewidth}
\emph{$1:$ .....................\\ $2:$ .....................}
\end{minipage}
\caption{Beispiel 2\,\cite{example}}
\label{fig:excample 2}
\end{figure}

\begin{figure}[h!]
\centering
\includegraphics[width=0.5\textwidth]{pictures/picture_excample.pdf}
\caption[Beispiel (Gro{\ss}e Abbildung zentriert)]{Beispiel (Gro{\ss}e Abbildung zentriert)\protect\\
\emph{$1$ ..................; $2$ ..................; $3$ ..............}}
	\label{fig:example 3}
\end{figure}

\section{Formeln}
\subsection{Formeln im Text}
Text $F_{xyz}\cdot P_{vw}=ZZZ$ Text....
\subsection{Formeln im Text integriert}
Text $\frac{P_{XY}}{Q_{VW}}=TTT$ Text......
\subsection{Formeln im Text dargestellt mit \glqq nicefracn\grqq}
Text $\nicefrac{P_{XY}}{Q_{VW}}=TTT$ Text......

\subsection{Nummerierte Formeln mit Erl\"auterung}
Bitte benutzen Sie den \glqq dfrac\grqq \,Befehl. Dieser Befehl ergibt ein besseres Schriftbild insbesondere bei doppelten Br\"uchen.
\begin{equation}
\label{equ:formula with explanation}
P_e= \dfrac{H_U \cdot \rho_L} {\lambda \cdot L_{min} +1} + \lambda_L \cdot \eta_e \cdot V_H \cdot \dfrac{T_U}{T_A} \cdot n
\end{equation}

\begin{align*}
&F &=& \text{ \ \ Kraft }&N&\\
&m &=& \text{ \ \ Masse }&kg&\\
&a &=& \text{ \ \ Beschleunigung }&\nicefrac{m}{s^2}&\\
\end{align*}

\newpage

\section{Tabellen}
Die Bezeichnung und Nummerierung kommt bei Tabellen \"uber die Tabelle.
\begin{table}[h]
	\centering
	\caption{Tabelle}
	\vspace{0.2cm}
		\begin{tabular}[t]{lcc}
		\bfseries{Text} & \bfseries{Text}  & \bfseries{Text}\\ \hline
		Text & $10$ & $40$ \\
		Text & $10$ & $80$ \\
		Text & $10$ & $50$ \\ \hline
		Text & $10$ & $100$ 
		\normalfont
		\end{tabular}
	\label{tab:tab1}
\end{table}

\newpage



