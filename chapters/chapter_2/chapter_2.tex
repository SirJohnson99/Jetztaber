\chapter{Grundlagen}


TEXT



\section{Definition Schall}



TEXT






Die Bezeichnung und Nummerierung kommt bei Tabellen \"uber die Tabelle.
\begin{table}[h]
	\centering
	\caption{Frequenzbereich in der Akkustik Quelle Lerch}
	\vspace{0.2cm}
		\begin{tabular}[t]{lcc}
		\bfseries{Frequenzbereich} & \bfseries{Bezeichnung} \\ \hline
		0 bis 20 Hz & 		$Infraschall$	 \\
		20 Hz bis 20 kHz & 	$Hörschall$ 	&\\
		20 kHz bis 1GHz & 	$Ultraschall$ 	&\\ 
		1 GHz bis 10 THz & 	$Hyperschall$ 	&\\  \hline
		\normalfont
		\end{tabular}
	\label{tab:tab1}
\end{table}


\section{Schallausbreitung in verschiedenen Medien}


TEXT


\subsection{Fluidschall}

\subsection{Körperschall}

\section{Komponenten einer Stereoanlage}

\subsection{Aufbau und Funktionsweise eines Exciters}

\section{Grundlagen der Plattenschwingung}

\section{Grundlagen der Schallabstrahlung}

\section{Werkstoffe für Exciterplatten}

\section{Simulationsstechnische Grundlagen}