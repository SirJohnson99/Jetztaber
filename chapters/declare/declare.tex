\usepackage{avant} 							%Avant als normale Schrift 
\usepackage[helvet]{sfmath} 					%serifenfreie Schrift f\"ur Mathmode


\renewcommand{\familydefault}{\sfdefault}
\usepackage[onehalfspacing]{setspace}			% Das brauchen eigentlich nur Leute, die
                                							% viel zu kleine R\UTF{00E4} verwenden oder keine
                                							% Korrekturzeichen beherrschen
\usepackage[utf8]{inputenc}
%\usepackage[latin1]{inputenc} 					% Zeichensatz, erm�glicht die direkte Eingabe von Umlauten im Editor
\usepackage[pdftex]{graphicx} 					% Einbindung von Grafiken (pdf, png, jpg)
\usepackage{float}            					         % bietet Option [H] f�r bombenfestes Verankern
\usepackage[ngerman]{babel}   					% Silbentrennung nach der neuen deutschen Rechtschreibung, z.B.: Sys-tem
\usepackage{amstext}          					% f�r Klartext via \text{} in Formeln
\usepackage{amsfonts} 
\usepackage{amsmath}        				         % f�r komplexere Formeln (Mengensymbole ...)
\usepackage{amssymb}        					 % f�r komplexere Formeln (Mengensymbole ...)
\usepackage{bm}           					         % bold math, f�r \bm{}
\usepackage{enumerate}        					% verbessert Aufz�hlungen
\usepackage[bottom]{footmisc} 					% Fussnoten am Seitenende
\usepackage{array}            					% f�r Tabellen: bindet tabular-Umgebung ein
\usepackage{algorithm}        					% f�r Algorithmen
\usepackage{algorithmic}      					% f�r Algorithmen
\usepackage{ntheorem}
\usepackage{theorem}
\usepackage{pdfpages}         					% f�r die Einbindung kompletter pdf-*Seiten*
\usepackage{parskip}          					% zw. Abs�tzen: eine knappe Leerzeile statt h�ngender Einz�ge
\usepackage[right]{eurosym}   					% Eurosymbol
\usepackage{xcolor}           					% farbiger Text
\usepackage{color}
\definecolor{OBlue}{rgb}{0.0,0.2,0.4}
\definecolor{OOrange}{rgb}{1.0,0.55,0.0}
\PassOptionsToPackage{hyphens}{url}    					% f�r \url{http://www}, Option hyp erlaubt auch Umbruch nach "-"
\usepackage[colorlinks=true,linkcolor=OBlue]{hyperref}			%Farbe Dunkelrot definieren
%\usepackage{colortbl}
%\usepackage{makeidx}          					% Package zur Indexerstellung
%\usepackage{multicol}         					% zur Indexerstellung in zwei Spalten
\usepackage[numbers, square]{natbib}   			% F�r \setlength{\bibsep}{3mm}; square macht eckige Klammern
\usepackage{wasysym}
\usepackage{pdfpages}						%Einf\"ugen von externen PDF-Dokumenten
\usepackage{epstopdf}
\usepackage[nice]{nicefrac}					%f\"ur Br\"uche im Text
\usepackage{fancyhdr}						%Kopf und Fu{\ss}zeile erm\"oglichen
\usepackage[T1]{fontenc}
\newcommand{\changefont}[3]{\fontfamily{#1} \fontseries{#2} \fontshape{#3} \selectfont}
\changefont{ptm}{m}{n}
\linespread {1.5}	

\sloppy  
\usepackage{titlesec}
\titleformat{\chapter}{\bf\Huge}{\thechapter\quad}{0em}{}                     					

%%%%%%%%%%%%%%%%
%Anpassung Inhaltsverzeichnis
%\usepackage{tocstyle}[2008/10/20]				%allwithdot oder noonewithdot verwenden
%\usetocstyle{noonewithdot} 					%ohne Punkte im Verzeichnis

\usepackage{verbatim}

% Gr��enanpassungen
\setlength{\unitlength}{1cm}
\setlength{\oddsidemargin}{0.3cm}
\setlength{\evensidemargin}{0.3cm}
\setlength{\textwidth}{16cm}
\setlength{\topmargin}{-1.2cm}
\setlength{\textheight}{23cm}
\columnsep 0.5cm

\makeindex 									% erstelle einen Index bzw. ein Sachverzeichnis)
\hyphenation{Samm-lung-en Samm-lung Stau-beck-en Vor-na-me-in-i-ti-al % Ver-st\"ar-ker-aus-gang 
Nach-na-me Kurz-be-zeich-nung deutsch-spra-chige deutsch-sprachig Screen-shot Screen-shots schluss-end-lich Schluss-end-lich Make-In-dex Da-tei-name Da-tei-namen Ur-instinkt Ur-instinkte}